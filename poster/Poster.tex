%%%%%%%%%%%%%%%%%%%%%%%%%%%%%%%%%%%%%%%%%
% NIWeek 2014 Poster by T. Reveyrand
% www.microwave.fr
% http://www.microwave.fr/LaTeX.html
%%%%%%%%%%%%%%%%%%%%%%%%%%%%%%%%%%%%%%%%%

%----------------------------------------------------------------------------------------
%	PACKAGES AND OTHER DOCUMENT CONFIGURATIONS
%----------------------------------------------------------------------------------------

\documentclass[a0paper,portrait]{baposter}

\usepackage[font=small,labelfont=bf]{caption}
\usepackage{booktabs}
\usepackage{relsize}

\usepackage{amsmath,amsfonts,amssymb,amsthm}
\usepackage{eqparbox}
\usepackage{textcomp}

\usepackage{tikz} % <-- important pour \background avec tikzpicture

\graphicspath{{figures/}}

%----------------------------------------------------------------------------------------
%	COULEURS (palette claire et neutre)
%----------------------------------------------------------------------------------------

\definecolor{bordercol}{RGB}{80,80,80}        % Bordures
\definecolor{esilvcol}{RGB}{206,16,82}        % ESILV color #CE1052
\definecolor{headercol1}{RGB}{206,16,82}      % Header gauche (ESILV)
\definecolor{headercol2}{RGB}{206,16,82}      % Header droite (ESILV)
\definecolor{headerfontcol}{RGB}{255,255,255}% Texte header (blanc)
\definecolor{boxcolor}{RGB}{250,250,248}      % Contenu (blanc cassé)

% Couleur de fond de la page
\definecolor{bgcol}{RGB}{252,252,250}         % Fond global très clair      

\begin{document}

%----------------------------------------------------------------------------------------
%	FOND DE PAGE CLAIR (remplace l'image "background")
%----------------------------------------------------------------------------------------
\background{
\begin{tikzpicture}[remember picture,overlay]
  \fill[bgcol] (current page.south west) rectangle (current page.north east);
\end{tikzpicture}
}

%----------------------------------------------------------------------------------------
%	POSTER LAYOUT
%----------------------------------------------------------------------------------------
\begin{poster}{
grid=false,
borderColor=bordercol,
headerColorOne=headercol1,
headerColorTwo=headercol2,
headerFontColor=headerfontcol,
boxColorOne=boxcolor,
headershape=roundedright,
headerfont=\Large\sf\bf,
textborder=rectangle,
background=user,
headerborder=open,
boxshade=plain
}
{\includegraphics[width=0.1\textwidth]{esilv.png}}
%
%----------------------------------------------------------------------------------------
%	TITLE AND AUTHOR NAME
%----------------------------------------------------------------------------------------
%
{ \centering \bf  \huge {Energy Aware Computing Continuum} \\  \Large \it Energy Modeling Across the Cloud-Edge Computing Continuum, Comparative Synthesis and practical guideline}
{\centering \vspace{0.3em} \smaller Théo HARDY$^1$, Farah Ait Salaht$^1$, WLADDIMIRO Daniel$^1$  \\
\smaller $^1$\it {ESILV / DVRC} \\}  
{\includegraphics[width=0.15\textwidth]{dvrc.png}}

%----------------------------------------------------------------------------------------
%	INTRODUCTION
%----------------------------------------------------------------------------------------
%----------------------------------------------------------------------------------------
%	REVISED INTRODUCTION: THE ENERGY VS. PERFORMANCE GAP
%----------------------------------------------------------------------------------------
\headerbox{Introduction}{name=introduction,column=0,row=0, span=3}{
\begin{minipage}{0.65\linewidth}
\begin{itemize}
\item \textbf{The Optimization Challenge:}
\vspace{-0.2cm}
  \begin{itemize}
  \item The \textbf{Cloud-Edge Computing Continuum (CECC)} enables flexible task placement.
  \vspace{-0.5cm}
  \item Orchestration must balance performance (latency, locality) with minimal energy costs.
  \end{itemize}
\vspace{-0.3cm}
\item \textbf{Taxonomy and Synthesis:}
\vspace{-0.2cm}
  \begin{itemize}
  \item Existing energy metrics are highly fragmented.
  \vspace{-0.1cm}
  \item We establish a \textbf{comprehensive taxonomy} of operational energy formulations.
  \vspace{-0.1cm}
  \item Models are analyzed by device scope, abstraction level, and measurement methodology.
  \end{itemize}
\vspace{-0.3cm}
\item \textbf{Real-World Benchmarking:}
\vspace{-0.2cm}
  \begin{itemize}
  \item We created \textbf{real-world experiment on Google Cloud Platform (GCP)}.
  \vspace{-0.1cm}
  \item We evaluate using a multi-component Data Stream Processing (DSP) application.
  \end{itemize}
\end{itemize}
\end{minipage}
\hfill
\begin{minipage}{0.33\linewidth}
\centering
\includegraphics[width=\linewidth, trim={1.2cm 1.5cm 1.2cm 1.5cm}, clip]{Cloud_Fog_Edge_scheme}
\end{minipage}
}

%----------------------------------------------------------------------------------------
%	Energy Formulation 
% or 
% Comparative synthesis
%----------------------------------------------------------------------------------------
\headerbox{Calibration}{name=calibration,column=0,below=introduction}{
LSNA calibration algorithm consists of \textbf{3 steps} at each RF frequency:
\begin{enumerate}
\item A relative VNA calibration creates an error-term matrix related to ports 1 and 2:
\begin{equation*}
\begin{pmatrix} a_ 1 \\  b_1 \\  a_2 \\ b_2 \end{pmatrix}=K\begin{bmatrix} 1 & \beta_1  & 0 & 0\\  \gamma_1 &  \delta_1  & 0 & 0 \\ 0 & 0 & \alpha_2 & \beta_2 \\ 0 & 0 &  \gamma_2 &  \delta_2 \end{bmatrix}.\begin{pmatrix} r_ 1 \\  r_2 \\  r_3 \\ r_4 \end{pmatrix}
\label{eq:cal_2_ports}
\end{equation*}

\item The power calibration gives $|K|$
\item The phase calibration yields $\arg\{K\}$
\end{enumerate}

Power and phase calibration are performed at an auxiliary reference plane ($P_{aux}$) after its own 1-port SOL coaxial calibration:

\begin{equation*}
\begin{pmatrix} a_{aux} \\  b_{aux} \end{pmatrix}=K_{aux}\begin{bmatrix} 1 & \beta_{aux} \\  \gamma_{aux} &  \delta_{aux} \end{bmatrix}.\begin{pmatrix} r_1 \\  r_2  \end{pmatrix}
\label{eq:cal_port_aux}
\end{equation*}

\begin{center}
\includegraphics[width=0.7\linewidth]{CALIBRATION.pdf}
\end{center}

\textbf{$\Rightarrow$ Power} calibration at $P_{aux}$ reference plane requires the connection of a power sensor. According to the measured value, in $dBm$, we can calculate $|K_{aux}|$ such as:
\begin{equation*}
|K_{aux}|=\left|{\frac{10^{(Power-10)/20}}{ r_1 + \beta_{aux} . r_2}}\right|
\label{eq:cal_power}
\end{equation*}

\textbf{$\Rightarrow$ Phase} calibration at $P_{aux}$ is performed by connecting a direct receiver (e.g. $r_3$) at $P_{aux}$:
\begin{equation*}
\arg\{K_{aux}\}=\arg\left\{{\frac{r_3}{ r_1 + \beta_{aux} . r_2}}\right\}
\label{eq:cal_phase}
\end{equation*}

\textbf{$\Rightarrow$ Reciprocity} transfers the absolute calibration from $P_{aux}$ to ports 1 and 2 ($P1$ and $P2$):
\begin{equation*}
K=\pm\sqrt{1/Det\{[M]\}}
\label{eq:reciprocity_1}
\end{equation*}
with
\begin{equation*}
M=\begin{bmatrix} 1 & \beta_1  \\  \gamma_1 &  \delta_1 \end{bmatrix}. {\left [ K_{aux}.\begin{bmatrix} 1 & \beta_{aux}  \\  \gamma_{aux} &  \delta_{aux}  \end{bmatrix} \right]}^{-1}
\label{eq:reciprocity_2}
\end{equation*}
}

%----------------------------------------------------------------------------------------
%	Experimental Setup
%----------------------------------------------------------------------------------------
\headerbox{Experimental s}{name=instruments,span=2,column=1,row=1, below=introduction}{
\begin{center}
\resizebox{0.9\textwidth}{!}{\begin{minipage}{\textwidth}
\begin{tabular}{l l l l}
\toprule
\textbf{Name} & \textbf{Manufacturer} & \textbf{Receivers} & \textbf{Availability}\\
\midrule
MTA (requires two synchronized) & HP & Sampler & Discontinued \\
LSNA & Agilent & Sampler & Discontinued \\
PNA-X + Nonlinear option & Agilent & Mixer & \$\$ \\
ZVA + Nonlinear option & Rohde and Schwarz & Mixer &  \$\$ \\
SWAP X-402 & VTD & Sampler & Discontinued \\
\bottomrule
\end{tabular}
\end{minipage}}
\end{center}
}

%----------------------------------------------------------------------------------------
%	DSP and INFRA ARCHI
%----------------------------------------------------------------------------------------
\headerbox{Receiver: Mixer vs. Sampler}{name=receiver,span=2,column=1,row=1, below=instruments}{
\begin{center}
\includegraphics[width=1\linewidth]{RECEIVER.pdf}
\end{center}
}

%----------------------------------------------------------------------------------------
%	MEASUREMENT SETUP
%----------------------------------------------------------------------------------------
\headerbox{Measurement Setup for Envelope Tracking Application}{name=application,span=2,column=1,below=receiver}{
The setup includes \textbf{two LSNAs simultaneously}. One is dedicated to RF (sampler based downconversion), the other one samples directly the LF stimulus. The purpose is to investigate \textbf{low-frequencies $S_{22}$} of the DUT under RF large signal conditions.
\begin{center}
\includegraphics[width=\linewidth]{BENCH.pdf}
\small \textit{Low-frequency measurement of drain supply envelope-bandwidth impedance for supply-modulated PAs}
\end{center}
}

%----------------------------------------------------------------------------------------
%	CONCLUSION
%----------------------------------------------------------------------------------------
\headerbox{Conclusion}{name=conclusion,column=1,below=application,span=2}{
This new project will enable a new RF measurement capability by enabling an instrument that currently does not exist on the market. Some additional benefits include:
\vspace{-0.2cm}
\begin{itemize}
\item frequency range extension of NI RF instrument products currently available;
\vspace{-0.2cm}
\item sampler architecture offers a unique multi-scale time analysis possibility (e.g. signal and carrier domains);
\vspace{-0.2cm}
\item can be implemented with various ADCs and downconverters (e.g. THAs);
\vspace{-0.2cm}
\item 100\% LabVIEW environment;
\vspace{-0.2cm}
\item goal is to offer open-source LabVIEW software for user measurement flexibility.
\end{itemize}
}

%----------------------------------------------------------------------------------------
%	ACKNOWLEDGEMENTS
%----------------------------------------------------------------------------------------
\headerbox{Acknowledgements}{name=acknowledgements,column=0,below=conclusion, above=bottom,span=3}{
\smaller
This work is funded by National Instruments (Dr. Truchard) through a charitable donation. We would like to acknowledge DARPA (Dr. Greene) and ONR (Dr. Maki) for funding the initial part of this work under grant N00014-11-1-0931. \hfill \tiny \textit{Poster downloaded from} \textbf{www.microwave.fr}
}

\end{poster}

\end{document}